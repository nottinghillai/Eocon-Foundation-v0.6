%%%%%%%%%%%%%%%%%%%%%%%%%%%%%%%%%%%%%%%%%%%%%%%%%%%%%%%%%%%%%%%%%%%%%%%%%%%%%%%%
%% FOUNDATION CONSTITUTION TEMPLATE
%% Version: 1.0
%% Purpose: Standard template for Planner Agent constitution generation
%% 
%% INSTRUCTIONS FOR PLANNER AGENT:
%% 1. Replace all <PLACEHOLDER> tokens with actual values
%% 2. Remove or comment out sections not needed for specific project
%% 3. Add additional sections as required by project scope
%% 4. Maintain Article numbering consistency after modifications
%% 5. Update Table of Contents by recompiling
%%
%% PLACEHOLDER LEGEND:
%% <PROJECT-NAME>         - Name of the project/micro-app
%% <PROJECT-TYPE>         - Type: Micro-App, Constellation, Service, etc.
%% <VERSION>              - Semantic version (e.g., 1.0.0)
%% <AUTHOR>               - Primary author/owner
%% <ABSTRACT>             - Executive summary (2-3 sentences)
%% <PURPOSE-STATEMENT>    - Why this system exists
%% <DOMAIN>               - Business domain (e.g., Healthcare, Finance)
%% <TARGET-STACK>         - Technology stack (e.g., Swift/SwiftUI, Next.js)
%% <PARENT-CONSTITUTION>  - Parent constitution ID (for sub-constitutions)
%%%%%%%%%%%%%%%%%%%%%%%%%%%%%%%%%%%%%%%%%%%%%%%%%%%%%%%%%%%%%%%%%%%%%%%%%%%%%%%%

\documentclass[12pt,letterpaper]{article}

%===============================================================================
% PACKAGES - Standard Foundation Configuration
%===============================================================================
\usepackage[utf8]{inputenc}
\usepackage[margin=1in]{geometry}
\usepackage{graphicx}
\usepackage{amsmath}
\usepackage{amssymb}
\usepackage{enumitem}
\usepackage{tcolorbox}
\usepackage{fancyhdr}
\usepackage{titlesec}
\usepackage{xcolor}
\usepackage{tocloft}
\usepackage{hyperref}
\usepackage{longtable}
\usepackage{booktabs}

%===============================================================================
% COLOR DEFINITIONS - Foundation Standard Palette
%===============================================================================
\definecolor{primaryblue}{RGB}{0,102,204}
\definecolor{secondarygreen}{RGB}{0,153,76}
\definecolor{accentorange}{RGB}{255,102,0}
\definecolor{warningred}{RGB}{204,0,0}
\definecolor{neutralgray}{RGB}{128,128,128}

%===============================================================================
% HEADER/FOOTER CONFIGURATION
%===============================================================================
\pagestyle{fancy}
\fancyhf{}
\rhead{PROJECT-NAME}
\lhead{Constitutional Document}
\rfoot{Page \thepage}
\setlength{\headheight}{14.49998pt}
\addtolength{\topmargin}{-2.49998pt}

%===============================================================================
% TITLE FORMATTING
%===============================================================================
\titleformat{\section}
  {\normalfont\Large\bfseries\color{primaryblue}}{\thesection}{1em}{}
\titleformat{\subsection}
  {\normalfont\large\bfseries\color{secondarygreen}}{\thesubsection}{1em}{}
\titleformat{\subsubsection}
  {\normalfont\normalsize\bfseries\color{neutralgray}}{\thesubsubsection}{1em}{}

%===============================================================================
% TABLE OF CONTENTS FORMATTING
%===============================================================================
\setlength{\emergencystretch}{3em}
\renewcommand{\cftsecleader}{\cftdotfill{\cftdotsep}}
\renewcommand{\cftsubsecleader}{\cftdotfill{\cftdotsep}}
\renewcommand{\cftsubsubsecleader}{\cftdotfill{\cftdotsep}}

\renewcommand{\cftsecfont}{\normalfont\bfseries\color{primaryblue}}
\renewcommand{\cftsecpagefont}{\normalfont\bfseries\color{primaryblue}}
\setlength{\cftsecnumwidth}{3em}
\setlength{\cftsecindent}{0em}

\renewcommand{\cftsubsecfont}{\normalfont\color{black}}
\renewcommand{\cftsubsecpagefont}{\normalfont\color{black}}
\setlength{\cftsubsecnumwidth}{4em}
\setlength{\cftsubsecindent}{1.5em}

\setlength{\cftbeforetoctitleskip}{1em}
\setlength{\cftaftertoctitleskip}{1.5em}
\setlength{\cftparskip}{0.3em}
\setlength{\cftbeforesecskip}{0.5em}

\renewcommand{\cftdotsep}{2}
\renewcommand{\cftdot}{$\cdot$}

%===============================================================================
% HYPERLINK CONFIGURATION
%===============================================================================
\hypersetup{
    colorlinks=true,
    linkcolor=primaryblue,
    filecolor=primaryblue,
    urlcolor=accentorange,
    citecolor=secondarygreen,
    pdftitle={PROJECT-NAME - Constitutional Document},
    pdfauthor={AUTHOR},
    pdfsubject={PROJECT-TYPE Constitution},
    pdfkeywords={foundation, constitution, DOMAIN},
    bookmarksnumbered=true,
    bookmarksopen=true,
    bookmarksopenlevel=2
}

%===============================================================================
% CUSTOM ENVIRONMENTS
%===============================================================================
% Requirement Box
\newtcolorbox{requirementbox}[1][]{
    colback=blue!5,
    colframe=primaryblue,
    title=#1,
    fonttitle=\bfseries
}

% Constraint Box
\newtcolorbox{constraintbox}[1][]{
    colback=orange!5,
    colframe=accentorange,
    title=#1,
    fonttitle=\bfseries
}

% Integration Box
\newtcolorbox{integrationbox}[1][]{
    colback=green!5,
    colframe=secondarygreen,
    title=#1,
    fonttitle=\bfseries
}

% Warning Box
\newtcolorbox{warningbox}[1][]{
    colback=red!5,
    colframe=warningred,
    title=#1,
    fonttitle=\bfseries
}

%===============================================================================
% DOCUMENT START
%===============================================================================
\begin{document}

%-------------------------------------------------------------------------------
% TITLE PAGE
%-------------------------------------------------------------------------------
\begin{titlepage}
    \centering
    \vspace*{2cm}

    {\Huge\bfseries PROJECT-NAME}\\[0.5cm]
    {\Large Constitutional Document}\\[0.3cm]
    {\large PROJECT-TYPE}\\[0.5cm]

    \vspace{1cm}

    {\Large\itshape PURPOSE-STATEMENT}\\[2cm]

    \vspace{1cm}

    %% METADATA SECTION
    \begin{tabular}{ll}
        \textbf{Version:} & VERSION \\
        \textbf{Date:} & \today \\
        \textbf{Author:} & AUTHOR \\
        \textbf{Domain:} & DOMAIN \\
        \textbf{Target Stack:} & TARGET-STACK \\
        %% OPTIONAL: Parent Constitution (for sub-constitutions)
        % \textbf{Parent:} & PARENT-CONSTITUTION \\
    \end{tabular}

    \vfill

    %% ABSTRACT
    {\large\bfseries Abstract}\\[0.5cm]
    \begin{tcolorbox}[colback=gray!10,colframe=primaryblue,width=0.9\textwidth]
    ABSTRACT-TEXT-HERE
    \end{tcolorbox}

\end{titlepage}

%-------------------------------------------------------------------------------
% TABLE OF CONTENTS
%-------------------------------------------------------------------------------
\renewcommand{\contentsname}{Table of Contents}
\renewcommand{\cfttoctitlefont}{\huge\bfseries\color{primaryblue}}
\tableofcontents
\newpage

%===============================================================================
% ARTICLE I: PURPOSE AND SCOPE
%===============================================================================
\section{Article I: Purpose and Scope}

\subsection{Section 1: Mission Statement}

%% INSTRUCTION: Define the core mission of this project/micro-app
%% This should answer: "Why does this system exist?"

MISSION-STATEMENT-HERE

\subsection{Section 2: Scope Definition}

\subsubsection{In Scope}

%% INSTRUCTION: List what IS included in this project
\begin{itemize}
    \item IN-SCOPE-ITEM-1
    \item IN-SCOPE-ITEM-2
    \item IN-SCOPE-ITEM-3
    %% Add more items as needed
\end{itemize}

\subsubsection{Out of Scope}

%% INSTRUCTION: Explicitly list what is NOT included
\begin{itemize}
    \item OUT-OF-SCOPE-ITEM-1
    \item OUT-OF-SCOPE-ITEM-2
    %% Add more items as needed
\end{itemize}

\subsection{Section 3: Success Criteria}

%% INSTRUCTION: Define measurable success criteria
\begin{requirementbox}[Definition of Done]
\begin{enumerate}
    \item SUCCESS-CRITERION-1
    \item SUCCESS-CRITERION-2
    \item SUCCESS-CRITERION-3
    %% Add more criteria as needed
\end{enumerate}
\end{requirementbox}

%===============================================================================
% ARTICLE II: REQUIREMENTS
%===============================================================================
\section{Article II: Requirements}

\subsection{Section 1: Functional Requirements}

%% INSTRUCTION: List all functional requirements
%% Use FR-XXX format for traceability

\begin{longtable}{|p{2cm}|p{4cm}|p{6cm}|p{2cm}|}
\hline
\textbf{ID} & \textbf{Requirement} & \textbf{Description} & \textbf{Priority} \\
\hline
\endhead

FR-001 & FR-NAME-1 & FR-DESCRIPTION-1 & FR-PRIORITY-1 \\
\hline
FR-002 & FR-NAME-2 & FR-DESCRIPTION-2 & FR-PRIORITY-2 \\
\hline
%% Add more requirements as needed

\end{longtable}

\subsection{Section 2: Non-Functional Requirements}

%% INSTRUCTION: Define quality attributes and constraints

\subsubsection{Performance Requirements}

\begin{constraintbox}[Performance Constraints]
\begin{itemize}
    \item \textbf{Response Time:} RESPONSE-TIME-REQUIREMENT
    \item \textbf{Throughput:} THROUGHPUT-REQUIREMENT
    \item \textbf{Scalability:} SCALABILITY-REQUIREMENT
\end{itemize}
\end{constraintbox}

\subsubsection{Security Requirements}

\begin{itemize}
    \item SECURITY-REQUIREMENT-1
    \item SECURITY-REQUIREMENT-2
    %% Add more as needed
\end{itemize}

\subsubsection{Accessibility Requirements}

%% INSTRUCTION: Foundation mandates accessibility-first design
\begin{itemize}
    \item All UI elements SHALL have accessibility identifiers
    \item All interactive elements SHALL have accessibility labels and hints
    \item Minimum touch target: 44x44 points
    \item Dynamic Type support required
    \item VoiceOver compatibility required
    \item ADDITIONAL-A11Y-REQUIREMENT
\end{itemize}

\subsection{Section 3: Data Requirements}

%% INSTRUCTION: Define data models, schemas, and constraints

\subsubsection{Core Data Entities}

\begin{itemize}
    \item \textbf{ENTITY-1-NAME}: ENTITY-1-DESCRIPTION
    \item \textbf{ENTITY-2-NAME}: ENTITY-2-DESCRIPTION
    %% Add more entities as needed
\end{itemize}

\subsubsection{Data Constraints}

\begin{itemize}
    \item DATA-CONSTRAINT-1
    \item DATA-CONSTRAINT-2
\end{itemize}

%===============================================================================
% ARTICLE III: ARCHITECTURE
%===============================================================================
\section{Article III: Architecture}

\subsection{Section 1: Architectural Pattern}

%% INSTRUCTION: Specify the architectural pattern being used
%% Foundation default: MVVM-S (Model-View-ViewModel-Style)

This PROJECT-TYPE follows the \textbf{ARCHITECTURE-PATTERN} pattern.

\begin{tcolorbox}[colback=green!5,colframe=secondarygreen,title=Architecture Layers]
\begin{enumerate}
    \item \textbf{LAYER-1-NAME}: LAYER-1-DESCRIPTION
    \item \textbf{LAYER-2-NAME}: LAYER-2-DESCRIPTION
    \item \textbf{LAYER-3-NAME}: LAYER-3-DESCRIPTION
    %% Add more layers as needed
\end{enumerate}
\end{tcolorbox}

\subsection{Section 2: Component Structure}

%% INSTRUCTION: Define the major components/modules

\subsubsection{Core Components}

\begin{itemize}
    \item \textbf{COMPONENT-1-NAME}: COMPONENT-1-PURPOSE
    \item \textbf{COMPONENT-2-NAME}: COMPONENT-2-PURPOSE
    \item \textbf{COMPONENT-3-NAME}: COMPONENT-3-PURPOSE
\end{itemize}

\subsubsection{Component Dependencies}

%% INSTRUCTION: Document component relationships
\begin{verbatim}
DEPENDENCY-DIAGRAM-ASCII
\end{verbatim}

\subsection{Section 3: Technology Stack}

%% INSTRUCTION: Specify all technologies to be used

\begin{longtable}{|p{4cm}|p{4cm}|p{6cm}|}
\hline
\textbf{Layer} & \textbf{Technology} & \textbf{Justification} \\
\hline
\endhead

TECH-LAYER-1 & TECH-CHOICE-1 & TECH-JUSTIFICATION-1 \\
\hline
TECH-LAYER-2 & TECH-CHOICE-2 & TECH-JUSTIFICATION-2 \\
\hline
%% Add more as needed

\end{longtable}

%===============================================================================
% ARTICLE IV: USER EXPERIENCE
%===============================================================================
\section{Article IV: User Experience}

\subsection{Section 1: User Personas}

%% INSTRUCTION: Define target users

\begin{tcolorbox}[colback=blue!5,colframe=primaryblue,title=Primary Persona: PERSONA-1-NAME]
\begin{itemize}
    \item \textbf{Role:} PERSONA-1-ROLE
    \item \textbf{Goals:} PERSONA-1-GOALS
    \item \textbf{Pain Points:} PERSONA-1-PAIN-POINTS
    \item \textbf{Technical Level:} PERSONA-1-TECH-LEVEL
\end{itemize}
\end{tcolorbox}

%% Add more personas as needed

\subsection{Section 2: User Flows}

%% INSTRUCTION: Define primary user journeys

\subsubsection{Flow 1: USER-FLOW-1-NAME}

\begin{enumerate}
    \item USER-FLOW-1-STEP-1
    \item USER-FLOW-1-STEP-2
    \item USER-FLOW-1-STEP-3
    %% Add more steps as needed
\end{enumerate}

\textbf{Expected Outcome:} USER-FLOW-1-OUTCOME

\subsection{Section 3: UI/UX Standards}

%% INSTRUCTION: Define visual and interaction standards

\begin{itemize}
    \item \textbf{Design System:} DESIGN-SYSTEM
    \item \textbf{Color Scheme:} COLOR-SCHEME
    \item \textbf{Typography:} TYPOGRAPHY-SPEC
    \item \textbf{Spacing System:} SPACING-SYSTEM
\end{itemize}

%===============================================================================
% ARTICLE V: BUSINESS LOGIC
%===============================================================================
\section{Article V: Business Logic}

\subsection{Section 1: Core Business Rules}

%% INSTRUCTION: Define the fundamental business rules

\begin{requirementbox}[Business Rules]
\begin{enumerate}
    \item \textbf{BR-001}: BUSINESS-RULE-1
    \item \textbf{BR-002}: BUSINESS-RULE-2
    \item \textbf{BR-003}: BUSINESS-RULE-3
    %% Add more rules as needed
\end{enumerate}
\end{requirementbox}

\subsection{Section 2: Workflows}

%% INSTRUCTION: Define automated workflows and processes

\subsubsection{Workflow 1: WORKFLOW-1-NAME}

\textbf{Trigger:} WORKFLOW-1-TRIGGER

\textbf{Steps:}
\begin{enumerate}
    \item WORKFLOW-1-STEP-1
    \item WORKFLOW-1-STEP-2
    \item WORKFLOW-1-STEP-3
\end{enumerate}

\textbf{Output:} WORKFLOW-1-OUTPUT

\subsection{Section 3: State Management}

%% INSTRUCTION: Define state transitions and management

\begin{itemize}
    \item \textbf{Initial State:} INITIAL-STATE
    \item \textbf{Valid Transitions:} VALID-TRANSITIONS
    \item \textbf{Terminal States:} TERMINAL-STATES
\end{itemize}

%===============================================================================
% ARTICLE VI: INTEGRATION
%===============================================================================
\section{Article VI: Integration}

\subsection{Section 1: API Contracts}

%% INSTRUCTION: Define external and internal APIs

\subsubsection{Exposed APIs}

\begin{longtable}{|p{3cm}|p{2cm}|p{4cm}|p{4cm}|}
\hline
\textbf{Endpoint} & \textbf{Method} & \textbf{Description} & \textbf{Response} \\
\hline
\endhead

API-ENDPOINT-1 & API-METHOD-1 & API-DESCRIPTION-1 & API-RESPONSE-1 \\
\hline
%% Add more endpoints as needed

\end{longtable}

\subsubsection{Consumed APIs}

%% INSTRUCTION: List external dependencies

\begin{itemize}
    \item \textbf{EXTERNAL-API-1}: EXTERNAL-API-1-PURPOSE
    \item \textbf{EXTERNAL-API-2}: EXTERNAL-API-2-PURPOSE
\end{itemize}

\subsection{Section 2: Entanglement Points}

%% INSTRUCTION: Define integration with other micro-apps in constellation

\begin{integrationbox}[Constellation Integration]
\begin{itemize}
    \item \textbf{Upstream Dependencies:} UPSTREAM-DEPS
    \item \textbf{Downstream Consumers:} DOWNSTREAM-CONSUMERS
    \item \textbf{Shared Data:} SHARED-DATA
\end{itemize}
\end{integrationbox}

\subsection{Section 3: Data Contracts}

%% INSTRUCTION: Define data exchange formats

\begin{verbatim}
DATA-CONTRACT-SCHEMA
\end{verbatim}

%===============================================================================
% ARTICLE VII: TESTING REQUIREMENTS
%===============================================================================
\section{Article VII: Testing Requirements}

\subsection{Section 1: Testing Strategy}

%% INSTRUCTION: Foundation mandates test-driven generation
%% Tests must be generated WITH code, not after

\begin{warningbox}[Foundation Testing Mandate]
\textbf{All code SHALL be generated with corresponding tests.}
\begin{itemize}
    \item Tests are generated SIMULTANEOUSLY with implementation
    \item Tests run IMMEDIATELY after generation
    \item Code is NOT complete until ALL tests pass
    \item Maximum 3 debug attempts before human intervention
\end{itemize}
\end{warningbox}

\subsection{Section 2: Test Types Required}

\begin{longtable}{|p{3cm}|p{4cm}|p{4cm}|p{2cm}|}
\hline
\textbf{Test Type} & \textbf{Scope} & \textbf{Tool/Framework} & \textbf{Coverage} \\
\hline
\endhead

Unit Tests & UNIT-TEST-SCOPE & Swift Testing & UNIT-COVERAGE-TARGET \\
\hline
Integration Tests & INTEGRATION-TEST-SCOPE & Swift Testing & INT-COVERAGE-TARGET \\
\hline
UI Tests & UI-TEST-SCOPE & XCUITest & UI-COVERAGE-TARGET \\
\hline
%% Add more test types as needed

\end{longtable}

\subsection{Section 3: Acceptance Criteria}

%% INSTRUCTION: Define what constitutes passing tests

\begin{enumerate}
    \item All unit tests pass with $\geq$ MIN-UNIT-COVERAGE\% coverage
    \item All integration tests pass
    \item All UI tests pass
    \item All accessibility tests pass
    \item Performance benchmarks met
    \item Security scan passes with no critical issues
\end{enumerate}

%===============================================================================
% ARTICLE VIII: IMPLEMENTATION ROADMAP
%===============================================================================
\section{Article VIII: Implementation Roadmap}

%% INSTRUCTION: This section defines the build strategy for the Planner Agent.
%% It outlines MVP scope, wave decomposition, and iteration phases.
%% The Planner Agent uses this to create discrete plans for the pipeline.

\subsection{Section 1: MVP Definition}

%% INSTRUCTION: Define the Minimum Viable Product - the smallest functional version
%% that delivers core value. This is what gets built FIRST.

\begin{requirementbox}[MVP Scope - What Ships First]
\textbf{MVP Goal:} MVP-GOAL-STATEMENT

\textbf{Core Features (Must Have for MVP):}
\begin{enumerate}
    \item MVP-FEATURE-1
    \item MVP-FEATURE-2
    \item MVP-FEATURE-3
    %% Add minimum features needed for MVP
\end{enumerate}

\textbf{Excluded from MVP (Deferred):}
\begin{itemize}
    \item DEFERRED-FEATURE-1 (Target: VERSION-X)
    \item DEFERRED-FEATURE-2 (Target: VERSION-Y)
    %% Features explicitly NOT in MVP
\end{itemize}

\textbf{MVP Success Criteria:}
\begin{itemize}
    \item MVP-SUCCESS-CRITERION-1
    \item MVP-SUCCESS-CRITERION-2
\end{itemize}
\end{requirementbox}

\subsection{Section 2: Wave Decomposition Strategy}

%% INSTRUCTION: Break the implementation into digestible "waves" for the execution pipeline.
%% Each wave should be completable in a single model context window.
%% Waves enable parallel execution where dependencies allow.

\begin{tcolorbox}[colback=green!5,colframe=secondarygreen,title=Wave Planning Guidelines]
\begin{itemize}
    \item Each wave should be \textbf{self-contained} with clear inputs/outputs
    \item Target \textbf{one-shot model execution} (minimize context switching)
    \item Waves should produce \textbf{testable artifacts}
    \item Independent waves can execute in \textbf{parallel}
    \item Dependent waves must specify \textbf{explicit prerequisites}
\end{itemize}
\end{tcolorbox}

\subsubsection{Wave Structure}

%% INSTRUCTION: Define the waves needed to build this constitution.
%% Format: Wave ID, Name, Description, Dependencies, Estimated Complexity

\begin{longtable}{|p{1.5cm}|p{3cm}|p{5cm}|p{2.5cm}|p{1.5cm}|}
\hline
\textbf{Wave} & \textbf{Name} & \textbf{Deliverables} & \textbf{Dependencies} & \textbf{Size} \\
\hline
\endhead

W1 & WAVE-1-NAME & WAVE-1-DELIVERABLES & None (Foundation) & WAVE-1-SIZE \\
\hline
W2 & WAVE-2-NAME & WAVE-2-DELIVERABLES & WAVE-2-DEPS & WAVE-2-SIZE \\
\hline
W3 & WAVE-3-NAME & WAVE-3-DELIVERABLES & WAVE-3-DEPS & WAVE-3-SIZE \\
\hline
W4 & WAVE-4-NAME & WAVE-4-DELIVERABLES & WAVE-4-DEPS & WAVE-4-SIZE \\
\hline
W5 & WAVE-5-NAME & WAVE-5-DELIVERABLES & WAVE-5-DEPS & WAVE-5-SIZE \\
\hline
%% Add more waves as needed. Size: S (small), M (medium), L (large)

\end{longtable}

\subsubsection{Wave Dependency Graph}

%% INSTRUCTION: ASCII representation of wave dependencies for Stages Agent

\begin{verbatim}
WAVE-DEPENDENCY-GRAPH-ASCII

Example:
    [W1: Data Models]
           |
    +------+------+
    |             |
[W2: API]    [W3: UI Core]
    |             |
    +------+------+
           |
    [W4: Integration]
           |
    [W5: Polish & Deploy]
\end{verbatim}

\subsection{Section 3: Iteration Phases}

%% INSTRUCTION: Define the release roadmap from MVP through future versions.
%% Each phase builds on the previous, adding capabilities incrementally.

\subsubsection{Phase 1: MVP (VERSION-MVP)}

\begin{constraintbox}[MVP Release]
\textbf{Timeline:} MVP-TIMELINE

\textbf{Scope:}
\begin{itemize}
    \item MVP-SCOPE-ITEM-1
    \item MVP-SCOPE-ITEM-2
    \item MVP-SCOPE-ITEM-3
\end{itemize}

\textbf{Waves Included:} W1, W2, W3 (adjust as needed)

\textbf{Exit Criteria:}
\begin{itemize}
    \item Core functionality operational
    \item Basic test coverage ($\geq$ 70\%)
    \item Can be deployed to staging environment
\end{itemize}
\end{constraintbox}

\subsubsection{Phase 2: Version 1.0 (VERSION-1-0)}

\begin{tcolorbox}[colback=blue!5,colframe=primaryblue,title=v1.0 Release]
\textbf{Timeline:} V1-TIMELINE

\textbf{Additions to MVP:}
\begin{itemize}
    \item V1-ADDITION-1
    \item V1-ADDITION-2
    \item V1-ADDITION-3
\end{itemize}

\textbf{Waves Included:} W4, W5 (adjust as needed)

\textbf{Exit Criteria:}
\begin{itemize}
    \item Full feature set operational
    \item Test coverage ($\geq$ 80\%)
    \item Production-ready deployment
    \item Documentation complete
\end{itemize}
\end{tcolorbox}

\subsubsection{Phase 3: Future Iterations (VERSION-2-PLUS)}

%% INSTRUCTION: Outline future enhancements beyond v1.0

\begin{itemize}
    \item \textbf{v1.1:} V1-1-FEATURES
    \item \textbf{v1.2:} V1-2-FEATURES
    \item \textbf{v2.0:} V2-0-FEATURES (major iteration)
\end{itemize}

\subsection{Section 4: Critical Path Analysis}

%% INSTRUCTION: Identify the longest dependency chain and potential bottlenecks.
%% This helps the Stages Agent optimize parallel execution.

\begin{warningbox}[Critical Path]
\textbf{Longest Dependency Chain:}
\begin{verbatim}
CRITICAL-PATH-SEQUENCE
Example: W1 → W2 → W4 → W5 (estimated: X days)
\end{verbatim}

\textbf{Bottlenecks:}
\begin{itemize}
    \item BOTTLENECK-1: BOTTLENECK-1-MITIGATION
    \item BOTTLENECK-2: BOTTLENECK-2-MITIGATION
\end{itemize}

\textbf{Parallelization Opportunities:}
\begin{itemize}
    \item PARALLEL-OPP-1 (e.g., W2 and W3 can run simultaneously)
    \item PARALLEL-OPP-2
\end{itemize}
\end{warningbox}

\subsection{Section 5: Risk Assessment}

%% INSTRUCTION: Identify implementation risks and mitigation strategies.
%% Helps the Optimization Agent make informed decisions.

\begin{longtable}{|p{3cm}|p{2cm}|p{2cm}|p{5cm}|}
\hline
\textbf{Risk} & \textbf{Likelihood} & \textbf{Impact} & \textbf{Mitigation} \\
\hline
\endhead

RISK-1-NAME & RISK-1-LIKELIHOOD & RISK-1-IMPACT & RISK-1-MITIGATION \\
\hline
RISK-2-NAME & RISK-2-LIKELIHOOD & RISK-2-IMPACT & RISK-2-MITIGATION \\
\hline
RISK-3-NAME & RISK-3-LIKELIHOOD & RISK-3-IMPACT & RISK-3-MITIGATION \\
\hline
%% Likelihood: Low, Medium, High
%% Impact: Low, Medium, High, Critical

\end{longtable}

\subsection{Section 6: Planner Agent Directives}

%% INSTRUCTION: Specific instructions for the Planner Agent when processing this constitution.

\begin{integrationbox}[Planner Agent Instructions]
\textbf{Plan Generation Rules:}
\begin{enumerate}
    \item Generate one plan folder per wave (plan1/ for W1, etc.)
    \item Each wave plan must include corresponding test specifications
    \item Respect dependency order when numbering plans
    \item Mark independent waves for parallel execution in stages.md
    \item Include decision\_log.md in each plan folder
\end{enumerate}

\textbf{Wave Sizing Guidelines:}
\begin{itemize}
    \item \textbf{Small (S):} Single component, $<$500 lines, 1-2 files
    \item \textbf{Medium (M):} Multiple components, 500-1500 lines, 3-5 files
    \item \textbf{Large (L):} Complex feature, 1500+ lines, 5+ files (consider splitting)
\end{itemize}

\textbf{Test Requirements per Wave:}
\begin{itemize}
    \item Unit tests for all new functions/methods
    \item Integration tests for cross-component interactions
    \item UI tests for user-facing changes (if applicable)
\end{itemize}

\textbf{Special Considerations:}
\begin{itemize}
    \item PLANNER-DIRECTIVE-1
    \item PLANNER-DIRECTIVE-2
\end{itemize}
\end{integrationbox}

%===============================================================================
% ARTICLE IX: DEPLOYMENT
%===============================================================================
\section{Article IX: Deployment}

\subsection{Section 1: Deployment Strategy}

%% INSTRUCTION: Define how the system will be deployed

\begin{itemize}
    \item \textbf{Strategy:} DEPLOYMENT-STRATEGY
    \item \textbf{Environment:} DEPLOYMENT-ENVIRONMENT
    \item \textbf{Frequency:} DEPLOYMENT-FREQUENCY
\end{itemize}

\subsection{Section 2: Prerequisites}

%% INSTRUCTION: Define what must be true before deployment

\begin{enumerate}
    \item DEPLOYMENT-PREREQ-1
    \item DEPLOYMENT-PREREQ-2
    \item DEPLOYMENT-PREREQ-3
\end{enumerate}

\subsection{Section 3: Rollback Plan}

%% INSTRUCTION: Define rollback procedures

\begin{warningbox}[Rollback Procedure]
\begin{enumerate}
    \item ROLLBACK-STEP-1
    \item ROLLBACK-STEP-2
    \item ROLLBACK-STEP-3
\end{enumerate}
\end{warningbox}

%===============================================================================
% ARTICLE X: MAINTENANCE
%===============================================================================
\section{Article X: Maintenance}

\subsection{Section 1: Monitoring}

%% INSTRUCTION: Define monitoring and alerting

\begin{itemize}
    \item \textbf{Metrics:} KEY-METRICS
    \item \textbf{Alerts:} ALERT-CONDITIONS
    \item \textbf{Dashboards:} DASHBOARD-REQUIREMENTS
\end{itemize}

\subsection{Section 2: Support}

\begin{itemize}
    \item \textbf{SLA:} SLA-DEFINITION
    \item \textbf{Escalation Path:} ESCALATION-PATH
\end{itemize}

\subsection{Section 3: Amendment Process}

%% INSTRUCTION: Define how changes are made post-deployment

\begin{enumerate}
    \item Create Amendment document describing changes
    \item Reference affected Constitution sections
    \item Submit for review and approval
    \item Planner processes amendment
    \item Affected components regenerated
    \item Tests validate changes
    \item Deploy incrementally
\end{enumerate}

%===============================================================================
% APPENDICES
%===============================================================================
\appendix

\section{Appendix A: Glossary}

%% INSTRUCTION: Define domain-specific terms

\begin{longtable}{|p{4cm}|p{10cm}|}
\hline
\textbf{Term} & \textbf{Definition} \\
\hline
\endhead

TERM-1 & DEFINITION-1 \\
\hline
TERM-2 & DEFINITION-2 \\
\hline
%% Add more terms as needed

\end{longtable}

\section{Appendix B: References}

%% INSTRUCTION: List external references, documentation, standards

\begin{enumerate}
    \item REFERENCE-1
    \item REFERENCE-2
    %% Add more references as needed
\end{enumerate}

\section{Appendix C: Change Log}

%% INSTRUCTION: Track constitution changes

\begin{longtable}{|p{2cm}|p{2cm}|p{3cm}|p{6cm}|}
\hline
\textbf{Version} & \textbf{Date} & \textbf{Author} & \textbf{Changes} \\
\hline
\endhead

VERSION & \today & AUTHOR & Initial constitution \\
\hline
%% Add more changes as needed

\end{longtable}

%===============================================================================
% SIGNATURE BLOCK
%===============================================================================
\section*{Approval}

\vspace{1cm}

\begin{tcolorbox}[colback=gray!5,colframe=neutralgray]
\textbf{Constitution Approval}

\vspace{0.5cm}

This constitution has been reviewed and approved for execution by the Foundation pipeline.

\vspace{1cm}

\begin{tabular}{p{6cm}p{6cm}}
\hrulefill & \hrulefill \\
\textbf{Author/Owner} & \textbf{Date} \\
AUTHOR & \today \\
\end{tabular}

\vspace{1cm}

\begin{tabular}{p{6cm}p{6cm}}
\hrulefill & \hrulefill \\
\textbf{Reviewer} & \textbf{Date} \\
REVIEWER & \\
\end{tabular}
\end{tcolorbox}

%===============================================================================
% DOCUMENT END
%===============================================================================
\end{document}
