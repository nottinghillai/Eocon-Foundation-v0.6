\documentclass[12pt,letterpaper]{article}

% Packages
\usepackage[utf8]{inputenc}
\usepackage[margin=1in]{geometry}
\usepackage{graphicx}
\usepackage{amsmath}
\usepackage{amssymb}
\usepackage{enumitem}
\usepackage{tcolorbox}
\usepackage{fancyhdr}
\usepackage{titlesec}
\usepackage{xcolor}
\usepackage{tocloft}  % Enhanced TOC formatting - must be before hyperref
\usepackage{hyperref}  % Must be loaded last
\usepackage{graphicx}  % For including images
\usepackage{etoolbox}  % For better conditional handling
\newcommand{\includegraphicsifexists}[2][]{%
  \IfFileExists{#2}{%
    \includegraphics[#1]{#2}%
  }{%
    \fbox{\begin{minipage}{0.9\textwidth}
    \centering\vspace{1cm}
    \textbf{[Unified Framework Architecture Diagram]}\\
    \vspace{0.5cm}\textit{To generate this diagram, run:}\\
    \texttt{python tools/mermaid\_to\_image.py}\\
    \texttt{\quad .planning/Diagrams/unified-framework-architecture.mermaid}\\
    \texttt{\quad .planning/Constitution/unified-framework-architecture}\\
    \vspace{0.5cm}\textit{Or use mermaid.live to export manually.}
    \vspace{1cm}
    \end{minipage}}%
  }%
}

% Color definitions
\definecolor{primaryblue}{RGB}{0,102,204}
\definecolor{secondarygreen}{RGB}{0,153,76}
\definecolor{accentorange}{RGB}{255,102,0}

% Header/Footer
\pagestyle{fancy}
\fancyhf{}
\rhead{Foundation Framework}
\lhead{Constitutional Document}
\rfoot{Page \thepage}
\setlength{\headheight}{14.49998pt}
\addtolength{\topmargin}{-2.49998pt}

% Title formatting
\titleformat{\section}
  {\normalfont\Large\bfseries\color{primaryblue}}{\thesection}{1em}{}
\titleformat{\subsection}
  {\normalfont\large\bfseries\color{secondarygreen}}{\thesubsection}{1em}{}

% Enhanced Table of Contents Formatting
\setlength{\emergencystretch}{3em}  % Allow better line breaking
\renewcommand{\cftsecleader}{\cftdotfill{\cftdotsep}}  % Dotted leaders for sections
\renewcommand{\cftsubsecleader}{\cftdotfill{\cftdotsep}}  % Dotted leaders for subsections
\renewcommand{\cftsubsubsecleader}{\cftdotfill{\cftdotsep}}  % Dotted leaders for subsubsections

% Section formatting in TOC
\renewcommand{\cftsecfont}{\normalfont\bfseries\color{primaryblue}}  % Bold blue sections
\renewcommand{\cftsecpagefont}{\normalfont\bfseries\color{primaryblue}}  % Bold blue page numbers
\setlength{\cftsecnumwidth}{3em}  % Width for section numbers
\setlength{\cftsecindent}{0em}  % Section indent

% Subsection formatting in TOC
\renewcommand{\cftsubsecfont}{\normalfont\color{black}}  % Regular black subsections
\renewcommand{\cftsubsecpagefont}{\normalfont\color{black}}  % Regular black page numbers
\setlength{\cftsubsecnumwidth}{4em}  % Width for subsection numbers
\setlength{\cftsubsecindent}{1.5em}  % Subsection indent

% Subsubsection formatting in TOC
\renewcommand{\cftsubsubsecfont}{\normalfont\itshape\color{black}}  % Italic black subsubsections
\renewcommand{\cftsubsubsecpagefont}{\normalfont\color{black}}  % Regular black page numbers
\setlength{\cftsubsubsecnumwidth}{5em}  % Width for subsubsection numbers
\setlength{\cftsubsubsecindent}{3.5em}  % Subsubsection indent

% Spacing between TOC entries
\setlength{\cftbeforetoctitleskip}{1em}  % Space before TOC title
\setlength{\cftaftertoctitleskip}{1.5em}  % Space after TOC title
\setlength{\cftparskip}{0.3em}  % Space between entries
\setlength{\cftbeforesecskip}{0.5em}  % Space before sections
\setlength{\cftbeforesubsecskip}{0.2em}  % Space before subsections
\setlength{\cftbeforesubsubsecskip}{0.1em}  % Space before subsubsections

% Dot leader customization
\renewcommand{\cftdotsep}{2}  % Separation between dots
\renewcommand{\cftdot}{$\cdot$}  % Dot character

% Hyperlink styling for TOC
\hypersetup{
    colorlinks=true,
    linkcolor=primaryblue,
    filecolor=primaryblue,
    urlcolor=accentorange,
    citecolor=secondarygreen,
    pdftitle={Foundation Framework - Constitutional Document},
    pdfauthor={Emmanuel Theodore},
    pdfsubject={Meta-Development Operating System},
    pdfkeywords={constitutional programming, AI agents, software development, automation},
    bookmarksnumbered=true,
    bookmarksopen=true,
    bookmarksopenlevel=2,
    pdfstartview=FitH,
    pdfpagemode=UseOutlines
}

% Document
\begin{document}

% Title Page
\begin{titlepage}
    \centering
    \vspace*{2cm}

    {\Huge\bfseries The Foundation Framework\\[0.5cm]}
    {\Large Constitutional Document\\[0.3cm]}

    \vspace{1cm}

    {\Large\itshape A Meta-Development Operating System for\\
    Post-Vibe-Coding Software Engineering\\[2cm]}

    \vspace{2cm}

    {\large Version 1.0\\[0.3cm]}
    {\large \today\\[0.5cm]}
    {\large Author: Emmanuel Theodore\\[2cm]}

    \vfill

    {\large\bfseries Abstract}\\[0.5cm]
    \begin{tcolorbox}[colback=gray!10,colframe=primaryblue,width=0.9\textwidth]
    This document establishes the constitutional framework for Foundation, a revolutionary meta-development system that transforms natural language specifications into fully functional software constellations. Foundation represents the transition from traditional programming to constitutional programming, where applications are defined through structured documentation and instantiated through distributed, autonomous execution pipelines.
    \end{tcolorbox}

\end{titlepage}

% Table of Contents
\renewcommand{\contentsname}{Table of Contents}
\renewcommand{\cfttoctitlefont}{\huge\bfseries\color{primaryblue}}
\tableofcontents
\newpage

% Article I: Foundational Principles
\section{Article I: Foundational Principles}

\subsection{Section 1: Purpose and Vision}

Foundation exists to automate the automation process itself, establishing true recursion in software development. The framework enables:

\begin{enumerate}[label=\alph*)]
    \item \textbf{Speech-to-Company Capability}: The ability to instantiate complete business operations from natural language requirements
    \item \textbf{Constitutional Programming}: A paradigm where applications are defined through structured natural language documents rather than direct code manipulation
    \item \textbf{Distributed Democratization}: Decentralized compute infrastructure that rewards contribution and eliminates centralized environmental harm
    \item \textbf{Constellation Architecture}: Complete business ecosystems composed of specialized, entangled micro-applications
\end{enumerate}

\subsection{Section 2: Core Philosophy}

\begin{tcolorbox}[colback=secondarygreen!10,colframe=secondarygreen,title=The Three Pillars]
\begin{enumerate}
    \item \textbf{Learn Once, Document Once, Automate Forever}: Knowledge captured in structured documentation becomes perpetually executable
    \item \textbf{Humans Design, Machines Implement}: The separation of architectural thinking from mechanical implementation
    \item \textbf{True Recursion}: Not iteration, but recursion—automating the automation itself at every level
\end{enumerate}
\end{tcolorbox}

\subsection{Section 3: The Paradigm Shift}

Foundation recognizes that Large Language Models have not merely enhanced programming—they have created \textit{English as a programming language}. However, the optimal interface is not conversational prompting but \textbf{structured constitutional documentation}.

\begin{equation}
\text{Traditional:} \quad \text{Code} \rightarrow \text{Application}
\end{equation}

\begin{equation}
\text{Foundation:} \quad \text{Constitution} \rightarrow \text{Plans} \rightarrow \text{Constellation}
\end{equation}

% Article II: Architectural Framework
\section{Article II: Architectural Framework}

\subsection{Section 1: The Five-Folder Foundation}

Every Foundation project shall maintain the following canonical structure with enhanced organization:

\begin{enumerate}
    \item \textbf{Constitution/}: The source of truth—LaTeX documents defining what the system IS and DOES
    \begin{itemize}
        \item C0: Root Constitution (main application concept)
        \item C1, C2...CN: Sub-Constitutions (individual micro-apps)
    \end{itemize}
    
    \item \textbf{Plans/}: Enhanced structure with wave-based decomposition
    \begin{verbatim}
Plans/
+-- planX/
|   +-- planX.md           # Main plan document
|   +-- testX.md           # Test specifications
|   +-- decision_log.md    # Autonomous decisions
|   +-- waves/             # Wave decomposition
|       +-- wave1/
|       |   +-- subplan1.md
|       |   +-- subtest1.md
|       |   +-- unit_tests.md
|       +-- waveN/
|           +-- subplanN.md
|           +-- subtestN.md
|           +-- unit_tests.md
+-- stages.md              # Execution strategy (DAG)
    \end{verbatim}
    
    \item \textbf{Guides/}: Self-documentation created by LLMs
    \begin{itemize}
        \item Implementation decisions explained
        \item Historical patterns documented
        \item Used by Optimization Agent for learning
    \end{itemize}
    
    \item \textbf{Docs/}: External documentation, API references
    \begin{itemize}
        \item Auto-scraped resources
        \item Library documentation
        \item Used by Optimization Agent for context
    \end{itemize}
    
    \item \textbf{Design/}: Shared code and reusable components
    \begin{itemize}
        \item Sample applications
        \item Code patterns
        \item Used by Optimization Agent for best practices
    \end{itemize}
\end{enumerate}

\textbf{Key Enhancements}:
\begin{itemize}
    \item \textit{Wave-Based Decomposition}: Plans broken into digestible sub-plans for token efficiency
    \item \textit{Decision Logging}: All autonomous decisions documented with rationale
    \item \textit{Test Parity}: Every plan and wave has corresponding test specifications
    \item \textit{Execution Strategy}: stages.md provides dependency-aware parallel execution plan
\end{itemize}

\subsection{Section 2: The Constellation Model}

Applications shall not be monolithic but rather \textbf{constellations}—networks of specialized micro-applications working in concert.

\subsubsection{Constellation Composition}

A constellation consists of:

\begin{itemize}
    \item \textbf{Core Business Functions}: Universal micro-apps (Legal, Accounting, HR, Analytics, Customer Support, Marketing)
    \item \textbf{Industry-Specific Functions}: Specialized micro-apps (Fleet Management, Kitchen Display, Medical Records)
    \item \textbf{Custom Functions}: Bespoke applications unique to specific requirements
    \item \textbf{Entanglement Layer}: Integration infrastructure enabling inter-app communication
\end{itemize}

\subsubsection{Example: Trucking Company Constellation}

\begin{tcolorbox}[colback=blue!5,colframe=primaryblue]
\textbf{Required Micro-Apps:}
\begin{itemize}
    \item Client Application (customer-facing)
    \item Service Application (driver-facing)
    \item Dispatch System
    \item Fleet Management \& Maintenance
    \item Legal \& Compliance
    \item Finance \& Accounting
    \item Human Resources Management
    \item Customer Support
    \item Analytics Dashboard
    \item Marketing Platform
    \item Insurance Management
\end{itemize}
\end{tcolorbox}

\subsection{Section 3: The Unified Execution Pipeline}

Foundation operates through a six-phase pipeline, orchestrated by five specialized agents, each with specific responsibilities and clear input/output contracts. Figures~\ref{fig:unified-framework-part1}, \ref{fig:unified-framework-part2}, \ref{fig:unified-framework-part3a}, and \ref{fig:unified-framework-part3b} illustrate the complete architecture flow.

\begin{tcolorbox}[colback=orange!5,colframe=accentorange,title=Pipeline Overview]
\textbf{Speech} $\rightarrow$ \textbf{Phase 0: Constitution} $\rightarrow$ \textbf{Phase 1: Planning} $\rightarrow$ \textbf{Phase 2: Questions} $\rightarrow$ \textbf{Phase 3: Optimization} $\rightarrow$ \textbf{Phase 4: Stages} $\rightarrow$ \textbf{Phase 5: Execution} $\rightarrow$ \textbf{Constellation}
\end{tcolorbox}

\clearpage
\begin{figure}[p]
\centering
\vspace*{\fill}
\includegraphicsifexists[width=\textwidth,height=0.8\textheight,keepaspectratio]{unified-framework-architecture-part1.png}
\vspace*{\fill}
\caption[Phase 0-1: Constitution and Planning]{Speech input through Constitution Agent (Phase 0) and Planner Agent (Phase 1), showing transformation from speech to structured plans and wave decomposition.}
\label{fig:unified-framework-part1}
\end{figure}
\clearpage

\begin{figure}[p]
\centering
\vspace*{\fill}
\includegraphicsifexists[width=\textwidth,height=0.8\textheight,keepaspectratio]{unified-framework-architecture-part2.png}
\vspace*{\fill}
\caption[Phase 2-3: Questions and Optimization]{Question Agent (Phase 2) identifies ambiguities and Optimization Agent (Phase 3) applies historical learning to enhance plans and tests.}
\label{fig:unified-framework-part2}
\end{figure}
\clearpage

\begin{figure}[p]
\centering
\vspace*{\fill}
\includegraphicsifexists[width=\textwidth,height=0.8\textheight,keepaspectratio]{unified-framework-architecture-part3a.png}
\vspace*{\fill}
\caption[Phase 4-5A: Stages Agent and Setup]{Stages Agent (Phase 4) creates dependency-aware execution strategy, followed by Phase 5 setup with parallel wave identification and test-driven generation assignment.}
\label{fig:unified-framework-part3a}
\end{figure}

\begin{figure}[p]
\centering
\vspace*{\fill}
\includegraphicsifexists[width=\textwidth,height=0.8\textheight,keepaspectratio]{unified-framework-architecture-part3b.png}
\vspace*{\fill}
\caption[Phase 4-5B: Test Execution and Deployment]{Test execution with immediate validation, debug loops, and final constellation deployment through cross-wave integration.}
\label{fig:unified-framework-part3b}
\end{figure}
\clearpage

\subsubsection{Phase 0: Constitution Agent}

\textbf{Input}: Natural language (speech or text)\\
\textbf{Output}: Multiple Constitution documents (C0, C1, C2...CN)\\
\textbf{Agent}: Constitution Agent

\textbf{Process}:
\begin{enumerate}
    \item \textit{Speech Capture}: Audio recorded as wave file
    \item \textit{Speech-to-Text}: Whisper algorithm converts audio to text
    \item \textit{C0 Generation}: Root Constitution created defining:
    \begin{itemize}
        \item General application concept
        \item Overall business model
        \item Constellation structure
        \item High-level requirements
    \end{itemize}
    \item \textit{Sub-Constitution Generation}: Individual constitutions (C1, C2...CN) for each micro-app defining:
    \begin{itemize}
        \item Specific micro-app purpose
        \item Detailed requirements
        \item Integration specifications
        \item API contracts
    \end{itemize}
\end{enumerate}

\textbf{Human Checkpoint}: Constitutions reviewed and approved before proceeding.

\subsubsection{Phase 1: Planner Agent}

\textbf{Input}: Constitution documents (C0, C1, C2...CN)\\
\textbf{Output}: Structured plan folders with wave decomposition\\
\textbf{Agent}: Planner Agent

\textbf{Process}:
\begin{enumerate}
    \item Parse Constitution documents
    \item Create discrete plans for each micro-app (plan1/, plan2/...planN/)
    \item Generate test specifications (test1.md, test2.md...testN.md)
    \item Decompose each plan into \textbf{waves}—digestible sub-plans designed for one-shot model execution
    \item Create wave folders with sub-plans and sub-tests
\end{enumerate}

\textbf{File Structure Generated}:
\begin{verbatim}
Plans/
+-- plan1/
|   +-- plan1.md          # Main plan
|   +-- test1.md          # Test specifications
|   +-- waves/
|       +-- wave1/
|       |   +-- subplan1.md
|       |   +-- subtest1.md
|       +-- wave2/
|       |   +-- subplan2.md
|       |   +-- subtest2.md
|       +-- waveN/
|           +-- subplanN.md
|           +-- subtestN.md
\end{verbatim}

\subsubsection{Phase 2: Question Agent}

\textbf{Input}: All plan folders and files\\
\textbf{Output}: Questions.md + Decision Log\\
\textbf{Agent}: Question Agent

\textbf{Process}:
\begin{enumerate}
    \item Analyze all plans and waves for ambiguities
    \item Identify unclear requirements and missing details
    \item Generate comprehensive question list
    \item Output Questions.md document
    \item Log all identified ambiguities in decision\_log.md
\end{enumerate}

\textbf{Responsibility}: Ambiguity detection only—does not answer questions.

\subsubsection{Phase 3: Optimization Agent (Senior Engineer)}

\textbf{Input}: Questions.md + All plans + Codebase context\\
\textbf{Output}: Optimized plans + Enhanced tests + Decision log\\
\textbf{Agent}: Optimization Agent (acts as senior engineer)

\textbf{Process}:
\begin{enumerate}
    \item Review Questions.md from Phase 2
    \item Search historical patterns:
    \begin{itemize}
        \item Analyze Guides/ folder
        \item Review Design/ folder
        \item Study Docs/ folder
        \item Examine previous implementations
    \end{itemize}
    \item Apply self-resolution algorithm:
    \begin{itemize}
        \item If historical pattern found: apply it
        \item If not required: skip feature (prevent scope creep)
        \item If required: apply industry-standard best practice
    \end{itemize}
    \item Optimize all plans (plan1.md, plan2.md...planN.md)
    \item Enhance all tests (test1.md, test2.md...testN.md)
    \item Optimize all wave sub-plans
    \item Generate wave-specific unit tests
    \item Document all decisions in decision\_log.md with rationale
\end{enumerate}

\textbf{Output Enhancement}:
\begin{itemize}
    \item All plans optimized with best practices
    \item Comprehensive test coverage
    \item Clear decision rationale
    \item Historical learning applied
\end{itemize}

\subsubsection{Phase 4: Stages Agent}

\textbf{Input}: All optimized plans and waves\\
\textbf{Output}: stages.md (Dependency Graph + Execution Strategy)\\
\textbf{Agent}: Stages Agent

\textbf{Process}:
\begin{enumerate}
    \item Analyze all plan relationships
    \item Identify wave dependencies
    \item Build Directed Acyclic Graph (DAG)
    \item Calculate execution stages:
    \begin{itemize}
        \item Stage 1: Independent waves (parallel execution)
        \item Stage 2: Dependent waves (parallel within stage)
        \item Stage N: Final dependent waves
    \end{itemize}
    \item Identify critical path
    \item Mark independent waves (can execute anytime)
    \item Generate stages.md with complete execution strategy
\end{enumerate}

\textbf{Output Format (stages.md)}:
\begin{verbatim}
# Execution Stages

## Stage 1 (Parallel)
- plan1/wave1, plan2/wave1, plan3/wave1

## Stage 2 (Parallel)
- plan1/wave2, plan2/wave2, plan3/wave3

## Critical Path
plan1 → plan2 → plan3
\end{verbatim}

\subsubsection{Phase 5: Parallel Execution (Test-Driven Generation)}

\textbf{Input}: stages.md + All optimized plans and waves\\
\textbf{Output}: Complete Constellation (Deployed Business)\\
\textbf{Agents}: Multiple execution agents (A, B, C, D, E, F...)

\textbf{Execution Protocol}:

\textit{For Each Stage}:
\begin{enumerate}
    \item Assign independent waves to parallel agents
    \item Each agent processes ONE wave with isolated context
    \item \textbf{Test-Driven Generation Protocol}:
    \begin{enumerate}[label=\alph*)]
        \item Generate code + tests SIMULTANEOUSLY
        \item Run tests IMMEDIATELY after generation
        \item Validate test results
        \item If tests PASS: Wave complete $\rightarrow$ proceed
        \item If tests FAIL: Enter debug loop
        \begin{itemize}
            \item Debug BOTH code AND tests
            \item Maximum 3 attempts
            \item Re-run tests after each fix
            \item If still failing: Signal human intervention
        \end{itemize}
    \end{enumerate}
    \item Document implementation in Guides/
    \item Mark wave complete only when tests pass
\end{enumerate}

\textit{Cross-Wave Integration}:
\begin{itemize}
    \item All completed waves integrate
    \item End-to-end testing across waves
    \item Cross-app integration validation
    \item Final constellation verification
\end{itemize}

\textbf{Human-in-Loop Trigger}: Signaled when:
\begin{itemize}
    \item Debug loop exceeds maximum attempts
    \item Critical ambiguity cannot be resolved
    \item Integration conflicts detected
\end{itemize}

\textbf{Success Criteria}:
\begin{itemize}
    \item All waves pass their tests
    \item Cross-wave integration successful
    \item Complete constellation functional
    \item Original Constitution requirements met
\end{itemize}

\subsubsection{Phase 6: Deployment \& Amendments}

\textbf{Deployment}:
\begin{itemize}
    \item Complete constellation verified
    \item Security audit passed
    \item Performance criteria validated
    \item Compliance verification complete
\end{itemize}

\textbf{Human Checkpoint}: Final review and approval before deployment.

\textbf{Post-Deployment Amendments}:
\begin{itemize}
    \item Amendment documents describe changes
    \item Affected micro-apps identified
    \item Only changed components re-executed
    \item Incremental deployment of updates
    \item Amendment history maintained
\end{itemize}

\subsection{Section 4: Agent Specialization \& Responsibilities}

Foundation's unified pipeline is orchestrated by five specialized agents, each with distinct responsibilities, skills, and input/output contracts.

\subsubsection{Agent 1: Constitution Agent}

\begin{tcolorbox}[colback=orange!5,colframe=accentorange,title=Constitution Agent Specification]
\textbf{Primary Role}: Transform natural language into structured constitutional documents

\textbf{Skills Required}:
\begin{itemize}
    \item Business analysis and requirement extraction
    \item System design and architecture
    \item Natural language processing
    \item Constellation pattern recognition
\end{itemize}

\textbf{Input}: Raw speech (audio) or natural language text

\textbf{Output}: Multiple Constitution documents
\begin{itemize}
    \item C0: Root Constitution
    \item C1, C2...CN: Micro-app Constitutions
\end{itemize}

\textbf{Status}: Black box (iterative refinement process to be defined)
\end{tcolorbox}

\subsubsection{Agent 2: Planner Agent}

\begin{tcolorbox}[colback=green!5,colframe=secondarygreen,title=Planner Agent Specification]
\textbf{Primary Role}: Decompose Constitutions into executable wave-based plans

\textbf{Skills Required}:
\begin{itemize}
    \item Task decomposition and analysis
    \item Test specification generation
    \item Wave boundary identification
    \item Context optimization for one-shot execution
\end{itemize}

\textbf{Input}: Constitution documents (C0, C1...CN)

\textbf{Output}:
\begin{itemize}
    \item Plan folders (plan1/, plan2/...planN/)
    \item Test specifications (testX.md)
    \item Wave decompositions (waves/waveN/)
    \item Sub-plans and sub-tests for each wave
\end{itemize}

\textbf{Focus}: Digestible, token-efficient plan generation
\end{tcolorbox}

\subsubsection{Agent 3: Question Agent}

\begin{tcolorbox}[colback=purple!5,colframe=purple,title=Question Agent Specification]
\textbf{Primary Role}: Identify ambiguities and generate comprehensive question lists

\textbf{Skills Required}:
\begin{itemize}
    \item Ambiguity detection
    \item Requirement analysis
    \item Critical thinking
    \item Question formulation
\end{itemize}

\textbf{Input}: All plan folders, waves, and test specifications

\textbf{Output}:
\begin{itemize}
    \item Questions.md (comprehensive question list)
    \item decision\_log.md (initial ambiguity log)
\end{itemize}

\textbf{Critical Limitation}: This agent identifies questions only—it does NOT answer them
\end{tcolorbox}

\subsubsection{Agent 4: Optimization Agent (Senior Engineer)}

\begin{tcolorbox}[colback=red!5,colframe=red,title=Optimization Agent Specification]
\textbf{Primary Role}: Answer questions and optimize plans using historical learning

\textbf{Skills Required}:
\begin{itemize}
    \item Senior engineering expertise
    \item Historical pattern recognition
    \item Best practice application
    \item Test generation and enhancement
    \item Self-resolution decision-making
\end{itemize}

\textbf{Input}:
\begin{itemize}
    \item Questions.md from Question Agent
    \item All plans and waves
    \item Guides/ folder (historical patterns)
    \item Design/ folder (code patterns)
    \item Docs/ folder (external documentation)
\end{itemize}

\textbf{Output}:
\begin{itemize}
    \item Optimized plans (all planX.md files)
    \item Enhanced tests (all testX.md files)
    \item Optimized sub-plans and sub-tests
    \item Wave-specific unit tests
    \item Comprehensive decision\_log.md with rationale
\end{itemize}

\textbf{Self-Resolution Algorithm}:
\begin{enumerate}
    \item Search project history for patterns
    \item If pattern found: Apply historical approach
    \item If not required: Skip feature (prevent scope creep)
    \item If required: Apply industry-standard best practice
    \item Document ALL decisions with clear rationale
\end{enumerate}
\end{tcolorbox}

\subsubsection{Agent 5: Stages Agent}

\begin{tcolorbox}[colback=brown!5,colframe=brown,title=Stages Agent Specification]
\textbf{Primary Role}: Analyze dependencies and generate parallel execution strategy

\textbf{Skills Required}:
\begin{itemize}
    \item Dependency analysis
    \item Graph theory (DAG construction)
    \item Parallel execution planning
    \item Critical path identification
\end{itemize}

\textbf{Input}: All optimized plans and waves

\textbf{Output}:
\begin{itemize}
    \item stages.md (complete execution strategy)
    \item Dependency graph (DAG)
    \item Parallel execution stages
    \item Critical path identification
    \item Independent wave markers
\end{itemize}

\textbf{Optimization Goal}: Maximize parallel execution while respecting dependencies
\end{tcolorbox}

\subsection{Section 5: Key Innovations}

The unified framework introduces several critical innovations:

\begin{enumerate}
    \item \textbf{Speech-to-Constitution}: Seamless transformation from spoken requirements to structured documentation. Uses Whisper algorithm for conversion.
    
    \item \textbf{Wave-Based Decomposition}: Plans broken into digestible sub-units optimized for:
    \begin{itemize}
        \item Token efficiency (smaller context windows)
        \item One-shot model execution (higher success rate)
        \item Parallel processing (reduced total time)
        \item Context isolation (fewer conflicts)
    \end{itemize}
    
    \item \textbf{Autonomous Self-Resolution}: Optimization Agent learns from historical patterns to answer questions without human intervention, dramatically reducing clarification loops
    
    \item \textbf{Test-Driven Planning}: Tests generated during planning phase (not after implementation), ensuring:
    \begin{itemize}
        \item Clear success criteria before code generation
        \item Immediate validation post-generation
        \item Parallel debugging of code AND tests
    \end{itemize}
    
    \item \textbf{Decision Transparency}: All autonomous decisions logged with rationale, enabling:
    \begin{itemize}
        \item Human review and override
        \item Pattern learning for future projects
        \item Audit trail for compliance
        \item Continuous improvement of decision quality
    \end{itemize}
    
    \item \textbf{Dependency-Aware Parallel Execution}: Stages calculated automatically, enabling maximum parallelism while respecting dependencies
\end{enumerate}

% Article III: Testing Framework
\section{Article III: The Testing Backbone}

\subsection{Section 1: Testing as Foundation}

Automated testing is the \textit{sine qua non} of Foundation. Without robust testing, the autonomous execution pipeline cannot function.

\subsection{Section 2: Test Generation}

Tests shall be automatically generated by specialized LLMs trained for test creation, including:

\begin{itemize}
    \item \textbf{Unit Tests}: Isolated function/component validation
    \item \textbf{Integration Tests}: Cross-component interaction verification
    \item \textbf{UI Tests}: User interface and experience validation
    \item \textbf{End-to-End Tests}: Complete user flow testing
    \item \textbf{Performance Tests}: Load and stress testing
    \item \textbf{Security Tests}: Vulnerability scanning
\end{itemize}

\subsection{Section 3: Swift Testing Priority}

The Swift testing framework shall serve as the initial proof-of-concept, establishing:

\begin{itemize}
    \item Automated test generation from Plan specifications
    \item Screenshot capture and OCR analysis for UI validation
    \item Integration with Xcode testing infrastructure
    \item Documentation templates for teaching LLMs test patterns
\end{itemize}

\subsection{Section 4: Debug Loops}

Upon test failure, the system shall:

\begin{enumerate}
    \item Capture error messages, logs, and screenshots
    \item Feed failure data to debugging-specialized LLM
    \item Generate and apply fixes
    \item Re-run tests
    \item Iterate up to N attempts (configurable)
    \item Signal human intervention if unresolved
\end{enumerate}

% Article IV: Distributed Infrastructure
\section[Article IV]{Article IV: Distributed Infrastructure and Economics}

\subsection{Section 1: Decentralized Compute Network}

Foundation shall operate on a \textbf{distributed compute network} where individual contributors provide computational resources.

\subsection{Section 2: Node Participation}

Any device meeting minimum specifications may join as a compute node:

\begin{itemize}
    \item Personal computers (Mac, Windows, Linux)
    \item Servers
    \item Edge devices with sufficient capability
\end{itemize}

\subsection{Section 3: Proof of Useful Work}

Unlike traditional cryptocurrency mining (proof of arbitrary work), Foundation nodes perform \textbf{proof of useful work}:

\begin{itemize}
    \item Executing Plans (code generation)
    \item Running tests
    \item Generating documentation
    \item Computing embeddings
    \item Maintaining vector databases
\end{itemize}

\subsection{Section 4: Token Economics}

\begin{tcolorbox}[colback=accentorange!10,colframe=accentorange,title=Economic Model]
\textbf{Nodes earn tokens proportional to:}
\begin{itemize}
    \item Computational resources contributed
    \item Task complexity completed
    \item Quality of output (test pass rates)
    \item Uptime and reliability
\end{itemize}

\textbf{Tokens can be used to:}
\begin{itemize}
    \item Request constellation generation
    \item Access premium micro-apps from library
    \item Priority task execution
    \item Governance participation
\end{itemize}
\end{tcolorbox}

\subsection{Section 5: Environmental and Social Justice}

By distributing compute across the network, Foundation:

\begin{itemize}
    \item Eliminates need for massive centralized data centers
    \item Avoids environmental pollution concentrated in marginalized communities
    \item Democratizes AI infrastructure
    \item Provides passive income to participants
    \item Enables true ownership of the network
\end{itemize}

% Article V: The Planner
\section{Article V: The Planner System}

\subsection{Section 1: Role and Responsibility}

The Planner is the \textit{compiler} of Foundation—it translates constitutional documentation into executable task graphs.

\subsection{Section 2: Planner Functions}

\begin{enumerate}
    \item \textbf{Parse Constitution}: Extract requirements, architecture, features
    \item \textbf{Identify Dependencies}: Determine which tasks depend on others
    \item \textbf{Generate Task Graph}: Create directed acyclic graph (DAG) of tasks
    \item \textbf{Classify Parallelization}: Identify parallel vs. sequential execution paths
    \item \textbf{Select Templates}: Choose appropriate plan templates for target stack
    \item \textbf{Adapt Templates}: Customize templates to specific requirements
    \item \textbf{Generate Plan Files}: Create individual Plan.md for each task
    \item \textbf{Allocate Resources}: Distribute tasks to compute network
\end{enumerate}

\subsection{Section 3: Plan Template Library}

The Planner maintains templates for common stacks:

\begin{itemize}
    \item Swift/SwiftUI (iOS/macOS)
    \item Next.js/React (Web)
    \item Python/FastAPI (Backend)
    \item React Native (Mobile)
    \item Electron (Desktop)
\end{itemize}

\subsection{Section 4: Adaptive Planning}

When no template exists, the Planner shall:

\begin{enumerate}
    \item Analyze Constitution requirements
    \item Research target stack documentation
    \item Generate new template structure
    \item Validate template through proof-of-concept
    \item Add validated template to library
\end{enumerate}

This is \textbf{recursion in action}—the system improves its own capabilities.

% Article VI: Micro-App Library
\section{Article VI: The Micro-App Library}

\subsection{Section 1: Library Structure}

The Foundation Library consists of:

\begin{itemize}
    \item \textbf{Core/}: Universal business function micro-apps
    \item \textbf{Industry/}: Sector-specific micro-apps
    \item \textbf{Community/}: Contributed micro-apps
    \item \textbf{Custom/}: Bespoke micro-apps for specific constellations
\end{itemize}

\subsection{Section 2: Micro-App Standards}

Each micro-app in the library shall include:

\begin{enumerate}
    \item \textbf{Constitution.tex}: Defining purpose, architecture, features
    \item \textbf{API Schema}: Documented interfaces for entanglement
    \item \textbf{Data Contracts}: Defined data structures and flows
    \item \textbf{Test Suite}: Comprehensive automated tests
    \item \textbf{Documentation}: Usage guides and integration instructions
    \item \textbf{Version History}: Amendment log and changelog
\end{enumerate}

\subsection{Section 3: Composability}

Micro-apps are designed for plug-and-play composition:

\begin{equation}
\text{Constellation} = f(\text{Core}_1, \text{Core}_2, ..., \text{Industry}_1, \text{Industry}_2, ..., \text{Custom}_1, ...)
\end{equation}

\subsection{Section 4: The Marketplace}

A marketplace enables:

\begin{itemize}
    \item Discovery of existing micro-apps
    \item Community contribution and sharing
    \item Premium specialized micro-apps
    \item Rating and review system
    \item Version management
\end{itemize}

% Article VII: Entanglement Layer
\section{Article VII: The Entanglement Layer}

\subsection{Section 1: Purpose}

The Entanglement Layer enables micro-apps within a constellation to communicate, share data, and function as a cohesive system.

\subsection{Section 2: Communication Protocols}

Micro-apps communicate through:

\begin{itemize}
    \item RESTful APIs
    \item GraphQL interfaces
    \item WebSocket connections
    \item Message queues
    \item Event buses
    \item Local mesh networks (for distributed edge computing)
\end{itemize}

\subsection{Section 3: Data Contracts}

All inter-app data exchange is governed by explicit contracts defining:

\begin{itemize}
    \item Data schemas
    \item Validation rules
    \item Transformation requirements
    \item Security and encryption
    \item Access control
\end{itemize}

\subsection{Section 4: Integration Testing}

The Entanglement Layer includes:

\begin{itemize}
    \item Mock services for isolated testing
    \item Integration test suites
    \item End-to-end validation
    \item Performance monitoring
    \item Error handling and recovery
\end{itemize}

% Article VIII: Human-in-the-Loop
\section{Article VIII: Human-in-the-Loop Protocols}

\subsection{Section 1: Checkpoints}

Mandatory human review occurs at:

\begin{enumerate}
    \item Constitution approval (before planning)
    \item Final verification (before deployment)
    \item Amendment approval (before modification)
\end{enumerate}

\subsection{Section 2: Intervention Triggers}

The system signals for human intervention when:

\begin{itemize}
    \item Debug loops exceed maximum iterations
    \item Test failures cannot be automatically resolved
    \item Security vulnerabilities detected
    \item Integration conflicts arise
    \item Ambiguous requirements require clarification
\end{itemize}

\subsection{Section 3: Documentation of Interventions}

All human interventions shall be:

\begin{enumerate}
    \item Logged with timestamp and context
    \item Documented with problem description
    \item Annotated with solution explanation
    \item Fed back into training data
    \item Incorporated into Guides/
\end{enumerate}

This enables the system to learn and reduce future intervention needs.

% Article IX: Quality Assurance
\section{Article IX: Quality Assurance}

\subsection{Section 1: Multi-Level Verification}

Quality is ensured through:

\begin{enumerate}
    \item \textbf{Code-Level}: Linting, compilation, static analysis
    \item \textbf{Unit-Level}: Individual component testing
    \item \textbf{Integration-Level}: Cross-component testing
    \item \textbf{System-Level}: End-to-end constellation testing
    \item \textbf{Constitutional-Level}: Verification against original requirements
\end{enumerate}

\subsection{Section 2: Continuous Testing}

Tests run:

\begin{itemize}
    \item After each code generation
    \item After each merge
    \item On deployment
    \item Periodically in production (smoke tests)
    \item Before amendment application
\end{itemize}

\subsection{Section 3: Multi-Model Comparison}

For critical applications, the same Constitution may be implemented by multiple LLMs:

\begin{itemize}
    \item Claude implementation
    \item GPT implementation
    \item Gemini implementation
    \item Cursor implementation
\end{itemize}

Results are compared for:

\begin{itemize}
    \item Functional equivalence
    \item Performance differences
    \item Code quality variations
    \item Best practices adherence
\end{itemize}

% Article X: Orchestrator
\section{Article X: The Main Orchestrator}

\subsection{Section 1: Role}

The Main Orchestrator is the highest-level intelligence that:

\begin{enumerate}
    \item Receives customer requests
    \item Analyzes business requirements
    \item Determines appropriate constellation
    \item Coordinates multi-app generation
    \item Manages resource allocation
    \item Monitors progress
    \item Handles errors and escalations
\end{enumerate}

\subsection{Section 2: Constellation Determination}

When receiving "I need a trucking company," the Orchestrator:

\begin{enumerate}
    \item Analyzes requirements
    \item Identifies business type (Logistics/Trucking)
    \item Retrieves constellation template
    \item Customizes based on specific needs
    \item Determines which micro-apps to include
    \item Checks library for existing micro-apps
    \item Identifies gaps requiring new development
    \item Generates or retrieves Constitutions for each micro-app
\end{enumerate}

\subsection{Section 3: Parallel Coordination}

The Orchestrator manages:

\begin{itemize}
    \item Spawning multiple Planner instances
    \item Distributing tasks across compute network
    \item Monitoring progress of all micro-apps
    \item Coordinating integration
    \item Handling dependencies between apps
\end{itemize}

\subsection{Section 4: Scale Management}

At scale (20,000+ concurrent requests), the Orchestrator:

\begin{itemize}
    \item Load balances across infrastructure
    \item Prioritizes tasks based on customer tier
    \item Optimizes resource utilization
    \item Maintains quality standards
    \item Provides status updates
\end{itemize}

% Article XI: Documentation Standards
\section{Article XI: Documentation Standards}

\subsection{Section 1: Constitution Format}

All Constitutions shall follow LaTeX formatting with sections:

\begin{enumerate}
    \item \textbf{Abstract}: High-level summary
    \item \textbf{Purpose}: Why this system exists
    \item \textbf{Requirements}: What must be accomplished
    \item \textbf{Architecture}: How the system is structured
    \item \textbf{User Flows}: How users interact with the system
    \item \textbf{Business Logic}: Core functionality and rules
    \item \textbf{Integration}: How components/apps connect
    \item \textbf{Testing Criteria}: How success is measured
    \item \textbf{Deployment}: How the system goes live
    \item \textbf{Maintenance}: Ongoing care and updates
\end{enumerate}

\subsection{Section 2: Plan Format}

All Plan.md files shall include:

\begin{itemize}
    \item Task description and objectives
    \item Dependencies (what must complete first)
    \item Technical approach
    \item Acceptance criteria
    \item Test requirements
    \item Estimated complexity
    \item Target stack specifications
\end{itemize}

\subsection{Section 3: Guide Format}

All Guides shall document:

\begin{itemize}
    \item What was implemented
    \item How it was implemented
    \item Decisions made and rationale
    \item Challenges encountered
    \item Solutions applied
    \item Lessons learned
\end{itemize}

% Article XII: Versioning and Amendments
\section{Article XII: Versioning and Amendments}

\subsection{Section 1: Constitution Versions}

Each Constitution maintains:

\begin{itemize}
    \item Major version (breaking changes)
    \item Minor version (new features)
    \item Patch version (bug fixes)
\end{itemize}

\subsection{Section 2: Amendment Process}

To modify a deployed constellation:

\begin{enumerate}
    \item Create Amendment.tex describing changes
    \item Reference specific Constitution sections
    \item Justify the change
    \item Describe expected impact
    \item Submit for review
    \item Upon approval: Planner processes amendment
    \item Only affected micro-apps are regenerated
    \item Tests run on changed components
    \item Integration tests validate constellation
    \item Deploy incrementally
\end{enumerate}

\subsection{Section 3: Version Control Integration}

All changes are tracked through:

\begin{itemize}
    \item Git version control
    \item Amendment history in Constitution
    \item Changelog maintenance
    \item Rollback capability
\end{itemize}

% Article XIII: Security and Compliance
\section{Article XIII: Security and Compliance}

\subsection{Section 1: Security by Design}

All generated code shall:

\begin{itemize}
    \item Follow secure coding practices
    \item Include input validation
    \item Implement proper authentication
    \item Use encryption for sensitive data
    \item Follow principle of least privilege
    \item Include security tests
\end{itemize}

\subsection{Section 2: Compliance Frameworks}

Constellations can be configured for:

\begin{itemize}
    \item HIPAA (healthcare)
    \item SOC 2 (service organizations)
    \item GDPR (data protection)
    \item PCI DSS (payment processing)
    \item Industry-specific regulations
\end{itemize}

\subsection{Section 3: Audit Trail}

Complete audit trail maintained:

\begin{itemize}
    \item Who requested what
    \item What was generated
    \item When it was deployed
    \item What changes were made
    \item Who approved each stage
\end{itemize}

% Article XIV: Future Extensions
\section{Article XIV: Future Extensions and Research}

\subsection{Section 1: Speech-to-Constitution}

Future development includes direct speech-to-Constitution pipeline:

\begin{enumerate}
    \item Customer speaks requirements
    \item Speech-to-text transcription
    \item LLM generates Constitution draft
    \item Human reviews and approves
    \item Pipeline proceeds
\end{enumerate}

\subsection{Section 2: Autonomous Agents}

Agent systems that:

\begin{itemize}
    \item Propose amendments based on usage patterns
    \item Suggest optimizations
    \item Identify bugs proactively
    \item Recommend new features
    \item Score amendment proposals
\end{itemize}

\subsection{Section 3: Cross-Constellation Learning}

System learns from:

\begin{itemize}
    \item Patterns across similar constellations
    \item Common micro-app combinations
    \item Successful architectural patterns
    \item Failure modes and resolutions
\end{itemize}

\subsection{Section 4: Formal Verification}

Advanced verification including:

\begin{itemize}
    \item Mathematical proof of correctness
    \item Model checking
    \item Property-based testing
    \item Formal methods integration
\end{itemize}

% Article XV: Governance
\section{Article XV: Governance and Evolution}

\subsection{Section 1: Constitutional Amendments}

This Constitution itself may be amended through:

\begin{enumerate}
    \item Proposal submission with rationale
    \item Community review period
    \item Technical feasibility assessment
    \item Vote by token holders
    \item Implementation upon approval
\end{enumerate}

\subsection{Section 2: Standards Committee}

A technical standards committee shall:

\begin{itemize}
    \item Review proposed changes to core framework
    \item Maintain documentation standards
    \item Curate micro-app library
    \item Resolve technical disputes
    \item Guide architectural evolution
\end{itemize}

\subsection{Section 3: Open Source Principles}

Foundation operates on:

\begin{itemize}
    \item Open source core framework
    \item Transparent development
    \item Community contributions welcome
    \item Meritocratic advancement
    \item Shared ownership model
\end{itemize}

% Conclusion
\section{Conclusion}

This Constitution establishes Foundation as more than a development tool—it is a \textbf{meta-development operating system} that fundamentally changes how software is conceived, designed, and instantiated.

By separating the \textit{what} (Constitution) from the \textit{how} (implementation), Foundation enables:

\begin{itemize}
    \item True abstraction of complexity
    \item Democratization of software creation
    \item Distributed, sustainable infrastructure
    \item Rapid constellation deployment
    \item Continuous improvement through recursion
\end{itemize}

Foundation is not the future of programming—it is the \textbf{post-programming paradigm}, where natural language constitutions become the source code, and applications are merely instantiations of well-defined ideas.

\vspace{1cm}

\begin{center}
\textit{We hold these truths to be self-evident: that software should be defined by what it does, not how it's written; that automation should automate itself; and that the means of computation should belong to those who contribute to it.}
\end{center}

\end{document}
